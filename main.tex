\documentclass{beamer}
\usetheme{Warsaw}

\usepackage{algorithm}
%    Q-circuit version 2
%    Copyright (C) 2004  Steve Flammia & Bryan Eastin
%    Last modified on: 9/16/2011
%
%    This program is free software; you can redistribute it and/or modify
%    it under the terms of the GNU General Public License as published by
%    the Free Software Foundation; either version 2 of the License, or
%    (at your option) any later version.
%
%    This program is distributed in the hope that it will be useful,
%    but WITHOUT ANY WARRANTY; without even the implied warranty of
%    MERCHANTABILITY or FITNESS FOR A PARTICULAR PURPOSE.  See the
%    GNU General Public License for more details.
%
%    You should have received a copy of the GNU General Public License
%    along with this program; if not, write to the Free Software
%    Foundation, Inc., 59 Temple Place, Suite 330, Boston, MA  02111-1307  USA

% Thanks to the Xy-pic guys, Kristoffer H Rose, Ross Moore, and Daniel Müllner,
% for their help in making Qcircuit work with Xy-pic version 3.8.  
% Thanks also to Dave Clader, Andrew Childs, Rafael Possignolo, Tyson Williams,
% Sergio Boixo, Cris Moore, Jonas Anderson, and Stephan Mertens for helping us test 
% and/or develop the new version.

\usepackage{xy}
\xyoption{matrix}
\xyoption{frame}
\xyoption{arrow}
\xyoption{arc}

\usepackage{ifpdf}
\ifpdf
\else
\PackageWarningNoLine{Qcircuit}{Qcircuit is loading in Postscript mode.  The Xy-pic options ps and dvips will be loaded.  If you wish to use other Postscript drivers for Xy-pic, you must modify the code in Qcircuit.tex}
%    The following options load the drivers most commonly required to
%    get proper Postscript output from Xy-pic.  Should these fail to work,
%    try replacing the following two lines with some of the other options
%    given in the Xy-pic reference manual.
\xyoption{ps}
\xyoption{dvips}
\fi

% The following resets Xy-pic matrix alignment to the pre-3.8 default, as
% required by Qcircuit.
\entrymodifiers={!C\entrybox}

\newcommand{\bra}[1]{{\left\langle{#1}\right\vert}}
\newcommand{\ket}[1]{{\left\vert{#1}\right\rangle}}
    % Defines Dirac notation. %7/5/07 added extra braces so that the commands will work in subscripts.
\newcommand{\qw}[1][-1]{\ar @{-} [0,#1]}
    % Defines a wire that connects horizontally.  By default it connects to the object on the left of the current object.
    % WARNING: Wire commands must appear after the gate in any given entry.
\newcommand{\qwx}[1][-1]{\ar @{-} [#1,0]}
    % Defines a wire that connects vertically.  By default it connects to the object above the current object.
    % WARNING: Wire commands must appear after the gate in any given entry.
\newcommand{\cw}[1][-1]{\ar @{=} [0,#1]}
    % Defines a classical wire that connects horizontally.  By default it connects to the object on the left of the current object.
    % WARNING: Wire commands must appear after the gate in any given entry.
\newcommand{\cwx}[1][-1]{\ar @{=} [#1,0]}
    % Defines a classical wire that connects vertically.  By default it connects to the object above the current object.
    % WARNING: Wire commands must appear after the gate in any given entry.
\newcommand{\gate}[1]{*+<.6em>{#1} \POS ="i","i"+UR;"i"+UL **\dir{-};"i"+DL **\dir{-};"i"+DR **\dir{-};"i"+UR **\dir{-},"i" \qw}
    % Boxes the argument, making a gate.
\newcommand{\meter}{*=<1.8em,1.4em>{\xy ="j","j"-<.778em,.322em>;{"j"+<.778em,-.322em> \ellipse ur,_{}},"j"-<0em,.4em>;p+<.5em,.9em> **\dir{-},"j"+<2.2em,2.2em>*{},"j"-<2.2em,2.2em>*{} \endxy} \POS ="i","i"+UR;"i"+UL **\dir{-};"i"+DL **\dir{-};"i"+DR **\dir{-};"i"+UR **\dir{-},"i" \qw}
    % Inserts a measurement meter.
    % In case you're wondering, the constants .778em and .322em specify
    % one quarter of a circle with radius 1.1em.
    % The points added at + and - <2.2em,2.2em> are there to strech the
    % canvas, ensuring that the size is unaffected by erratic spacing issues
    % with the arc.
\newcommand{\measure}[1]{*+[F-:<.9em>]{#1} \qw}
    % Inserts a measurement bubble with user defined text.
\newcommand{\measuretab}[1]{*{\xy*+<.6em>{#1}="e";"e"+UL;"e"+UR **\dir{-};"e"+DR **\dir{-};"e"+DL **\dir{-};"e"+LC-<.5em,0em> **\dir{-};"e"+UL **\dir{-} \endxy} \qw}
    % Inserts a measurement tab with user defined text.
\newcommand{\measureD}[1]{*{\xy*+=<0em,.1em>{#1}="e";"e"+UR+<0em,.25em>;"e"+UL+<-.5em,.25em> **\dir{-};"e"+DL+<-.5em,-.25em> **\dir{-};"e"+DR+<0em,-.25em> **\dir{-};{"e"+UR+<0em,.25em>\ellipse^{}};"e"+C:,+(0,1)*{} \endxy} \qw}
    % Inserts a D-shaped measurement gate with user defined text.
\newcommand{\multimeasure}[2]{*+<1em,.9em>{\hphantom{#2}} \qw \POS[0,0].[#1,0];p !C *{#2},p \drop\frm<.9em>{-}}
    % Draws a multiple qubit measurement bubble starting at the current position and spanning #1 additional gates below.
    % #2 gives the label for the gate.
    % You must use an argument of the same width as #2 in \ghost for the wires to connect properly on the lower lines.
\newcommand{\multimeasureD}[2]{*+<1em,.9em>{\hphantom{#2}} \POS [0,0]="i",[0,0].[#1,0]="e",!C *{#2},"e"+UR-<.8em,0em>;"e"+UL **\dir{-};"e"+DL **\dir{-};"e"+DR+<-.8em,0em> **\dir{-};{"e"+DR+<0em,.8em>\ellipse^{}};"e"+UR+<0em,-.8em> **\dir{-};{"e"+UR-<.8em,0em>\ellipse^{}},"i" \qw}
    % Draws a multiple qubit D-shaped measurement gate starting at the current position and spanning #1 additional gates below.
    % #2 gives the label for the gate.
    % You must use an argument of the same width as #2 in \ghost for the wires to connect properly on the lower lines.
\newcommand{\control}{*!<0em,.025em>-=-<.2em>{\bullet}}
    % Inserts an unconnected control.
\newcommand{\controlo}{*+<.01em>{\xy -<.095em>*\xycircle<.19em>{} \endxy}}
    % Inserts a unconnected control-on-0.
\newcommand{\ctrl}[1]{\control \qwx[#1] \qw}
    % Inserts a control and connects it to the object #1 wires below.
\newcommand{\ctrlo}[1]{\controlo \qwx[#1] \qw}
    % Inserts a control-on-0 and connects it to the object #1 wires below.
\newcommand{\targ}{*+<.02em,.02em>{\xy ="i","i"-<.39em,0em>;"i"+<.39em,0em> **\dir{-}, "i"-<0em,.39em>;"i"+<0em,.39em> **\dir{-},"i"*\xycircle<.4em>{} \endxy} \qw}
    % Inserts a CNOT target.
\newcommand{\qswap}{*=<0em>{\times} \qw}
    % Inserts half a swap gate.
    % Must be connected to the other swap with \qwx.
\newcommand{\multigate}[2]{*+<1em,.9em>{\hphantom{#2}} \POS [0,0]="i",[0,0].[#1,0]="e",!C *{#2},"e"+UR;"e"+UL **\dir{-};"e"+DL **\dir{-};"e"+DR **\dir{-};"e"+UR **\dir{-},"i" \qw}
    % Draws a multiple qubit gate starting at the current position and spanning #1 additional gates below.
    % #2 gives the label for the gate.
    % You must use an argument of the same width as #2 in \ghost for the wires to connect properly on the lower lines.
\newcommand{\ghost}[1]{*+<1em,.9em>{\hphantom{#1}} \qw}
    % Leaves space for \multigate on wires other than the one on which \multigate appears.  Without this command wires will cross your gate.
    % #1 should match the second argument in the corresponding \multigate.
\newcommand{\push}[1]{*{#1}}
    % Inserts #1, overriding the default that causes entries to have zero size.  This command takes the place of a gate.
    % Like a gate, it must precede any wire commands.
    % \push is useful for forcing columns apart.
    % NOTE: It might be useful to know that a gate is about 1.3 times the height of its contents.  I.e. \gate{M} is 1.3em tall.
    % WARNING: \push must appear before any wire commands and may not appear in an entry with a gate or label.
\newcommand{\gategroup}[6]{\POS"#1,#2"."#3,#2"."#1,#4"."#3,#4"!C*+<#5>\frm{#6}}
    % Constructs a box or bracket enclosing the square block spanning rows #1-#3 and columns=#2-#4.
    % The block is given a margin #5/2, so #5 should be a valid length.
    % #6 can take the following arguments -- or . or _\} or ^\} or \{ or \} or _) or ^) or ( or ) where the first two options yield dashed and
    % dotted boxes respectively, and the last eight options yield bottom, top, left, and right braces of the curly or normal variety.  See the Xy-pic reference manual for more options.
    % \gategroup can appear at the end of any gate entry, but it's good form to pick either the last entry or one of the corner gates.
    % BUG: \gategroup uses the four corner gates to determine the size of the bounding box.  Other gates may stick out of that box.  See \prop.

\newcommand{\rstick}[1]{*!L!<-.5em,0em>=<0em>{#1}}
    % Centers the left side of #1 in the cell.  Intended for lining up wire labels.  Note that non-gates have default size zero.
\newcommand{\lstick}[1]{*!R!<.5em,0em>=<0em>{#1}}
    % Centers the right side of #1 in the cell.  Intended for lining up wire labels.  Note that non-gates have default size zero.
\newcommand{\ustick}[1]{*!D!<0em,-.5em>=<0em>{#1}}
    % Centers the bottom of #1 in the cell.  Intended for lining up wire labels.  Note that non-gates have default size zero.
\newcommand{\dstick}[1]{*!U!<0em,.5em>=<0em>{#1}}
    % Centers the top of #1 in the cell.  Intended for lining up wire labels.  Note that non-gates have default size zero.
\newcommand{\Qcircuit}{\xymatrix @*=<0em>}
    % Defines \Qcircuit as an \xymatrix with entries of default size 0em.
\newcommand{\link}[2]{\ar @{-} [#1,#2]}
    % Draws a wire or connecting line to the element #1 rows down and #2 columns forward.
\newcommand{\pureghost}[1]{*+<1em,.9em>{\hphantom{#1}}}
    % Same as \ghost except it omits the wire leading to the left. 

\usepackage[noend]{algpseudocode}

\setbeamertemplate{caption}[numbered]

% remove headlines at a certain frame

\makeatletter
    \newenvironment{withoutheadline}{
        \setbeamertemplate{headline}[default]
        \def\beamer@entrycode{\vspace*{-\headheight}}
    }{}
\makeatother


\setbeamertemplate{footline}{%
  \leavevmode%
  \hbox{\begin{beamercolorbox}[wd=.5\paperwidth,ht=4.5ex,dp=2.125ex,leftskip=.3cm plus1fill,rightskip=.3cm]{author in head/foot}%
    \usebeamerfont{author in head/foot}\insertshortauthor
  \end{beamercolorbox}%
  \begin{beamercolorbox}[wd=.5\paperwidth,ht=4.5ex,dp=2.125ex,leftskip=.3cm,rightskip=.3cm plus1fil]{title in head/foot}%
    \usebeamerfont{title in head/foot}
    \parbox{.45\paperwidth}{\inserttitle}
  \end{beamercolorbox}}%
  \vskip0pt%
}

% numbering references
\setbeamertemplate{bibliography item}{\insertbiblabel}

\title{A Quantum Algorithm for Finding Common Matches between Databases with Reliable Behavior}

\author{Khaled El-Wazan}

\institute{Department of Mathematics and Computer Science, Faculty of Science, Alexandria University, Egypt}

\date{}

\begin{document}

\begin{withoutheadline}
\begin{frame}
\maketitle

\end{frame}
\end{withoutheadline}



%\begin{withoutheadline}
\begin{frame}
\tableofcontents
\end{frame}
%\end{withoutheadline}


\section{Introduction}

\subsection*{Problem Statement}
\begin{frame}
\frametitle{Problem Statement}

\begin{definition}
Consider having a set $\mathcal{Z}$ of $\kappa\geq 2$ lists, $\mathcal{Z}=\{L_0,\cdots,L_{\kappa-1}\}$. Each list $L_j\in \mathcal{Z}$ is of \textcolor{red}{$N=2^n$} \textcolor{red}{unstructured entries}, which has an oracle $U_j$ that is being used to access those entries in $L_j$. Each entry $i\in L_j=\{0,1,\cdots,N-1\}$ in the list $L_j$ is \textcolor{red}{mapped to either $0$ or $1$} according to any certain property satisfied by $i$ in $L_j$, \textit{i.e.} $f_j:L_j\rightarrow\{0,1\}$. The common elements problem is stated as follows: find the entry $i\in L_j $  such that $ \forall L_j\in \mathcal{Z},~ f_j(i)=1$.
\end{definition}

\begin{example}

$f_0(x_0,x_1,x_2)=x_0x_1$, $~~solns=\{110,111\}$ \\ $f_1(x_0,x_1,x_2)=x_0x_1x_2 $,  $solns=\{111\}$

\end{example}


\end{frame}



\subsection*{Literature Review}
\begin{frame}

\frametitle{Literature Review}

\begin{itemize}


\item In 1998, Burhman \textit{et al.} introduced a quantum algorithm \cite{buhrman} find common entries between \textcolor{red}{remotely separated lists} in $\mathcal{O}(p\sqrt{N})$, with $p$ trials and $N=2^n$.

\item In 2012, Tulsi provided a quantum algorithm \cite{tulsi} to find a \textcolor{red}{single} common entry between \textcolor{red}{two lists} using Grover algorithm, in $\mathcal{O}(\sqrt{N})$. 

\item In 2013, Pang \textit{et al.} proposed a quantum algorithm \cite{pang} to find common entries between two sets stored in \textcolor{red}{classical memory}, in $\mathcal{O}(\sqrt{N^2/C})$, where $C$ is the number of common entries.

\end{itemize}
\end{frame}


\section{Amplitude Amplification}
\begin{frame}
\frametitle{Amplitude Amplification}
\begin{itemize}

\item In 1996, Grover proposed a \textcolor{red}{unique approach} \cite{grover} to find a \textcolor{red}{single item} in a database, in $\mathcal{O}(\sqrt{N})$.

\item Boyer \textit{et al.} later \textcolor{red}{generalized Grover}'s quantum search algorithm \cite{quant-tightbound-for-search,grover-younes} to fit the purpose of \textcolor{red}{finding multiple} solutions $M$ to the oracle, in $\mathcal{O}(\sqrt{N/M})$.

\item \textcolor{red}{Grover}'s algorithm amplifies solutions to an \textcolor{red}{extend}, \textit{i.e.} $1\leq M\leq 3N/4$ \cite{quant-tightbound-for-search,grover-younes}.

\item Younes \textit{et al}'s algorithm \cite{younes} is \textcolor{red}{reliable} in case of multiple matches, \textit{i.e.} $1\leq M\leq N$. 

\item Both run in $\mathcal{O}(\sqrt{N/M})$. 


\end{itemize}


\end{frame}


\begin{frame}

\begin{algorithm}[H]
\caption{Younes quantum search algorithm.\label{1st-algo}}
\begin{algorithmic}[1]

\State \textcolor{red}{Prepare} a quantum register of $n+1$ qubits to the state $\ket{0}^{n+1}$.
\item Apply the Hadamard gates on the register to create a \textcolor{red}{uniform superposition} $\frac{1}{\sqrt{N}}\sum_{l=0}^{N-1}{\vert l\rangle}$.
\State  \textcolor{red}{Iterate} over the following \textcolor{red}{$q=\frac{\pi}{2\sqrt{2}}\sqrt{\frac{N}{M}}$} steps:
\begin{enumerate}[1.]
\item Apply the \textcolor{red}{function $U_f$}, to mark the solutions with entanglement.
\item Apply the \textcolor{red}{ diffusion operator} 

\begin{equation*}
Y=(H^{\otimes n}\otimes I )(2\vert 0\rangle\langle 0\vert -I_{n+1})(H^{\otimes n}\otimes I).
\end{equation*}


\end{enumerate}

\State \textcolor{red}{Measure} the output.

\end{algorithmic}
\end{algorithm}

\end{frame}


\section{Constructing the Oracle $U_\hbar$}

\subsection*{Constructing the oracle $U_\hbar$ for two databases}

\begin{frame}
\frametitle{Common Entries Oracle Construction: Two Oracles}


\begin{figure}[H]
\begin{align*}
\Qcircuit @C=0.3em @R=.3em  {
\lstick{\mid x_0 \rangle}& \qw & \qw & \qw & \qw & \multigate{8}{U_A}   & \qw & \qw & \qw & \multigate{8}{U_B} & \qw & \qw & \multigate{8}{U_A} & \qw & \multigate{8}{U_B} & \qw &\qw\\
\lstick{\mid x_1 \rangle}& \qw & \qw & \qw & \qw & \ghost{U_A}  & \qw & \qw & \qw & \ghost{U_B} & \qw & \qw & \ghost{U_A}& \qw & \ghost{U_B}& \qw  & \qw\\
& . & . & . & . &  & .  & . & . &  & .  & . &  & . && .\\
& . & . & . & . &  & .  & . & . &  & .  & . &  & . && .\\
\lstick{\mid x_i \rangle} & \qw & \qw & \qw & \qw & \ghost{U_A} & \qw & \qw & \qw & \ghost{U_B} & \qw & \qw & \ghost{U_A}& \qw & \ghost{U_B}& \qw  & \qw\\
& . & . & . & . &  & .  & . & . &  &  . & . &   & .&& .\\
& . & . & . & . &  & .  & . & . &  &   .& . &   & .&& .\\
\lstick{\mid x_{n-2} \rangle}& \qw & \qw & \qw & \qw & \ghost{U_A}  & \qw & \qw & \qw & \ghost{U_B} & \qw & \qw & \ghost{U_A}& \qw & \ghost{U_B}& \qw  & \qw \\
\lstick{\mid x_{n-1} \rangle}& \qw & \qw & \qw & \qw & \ghost{U_A}  & \qw & \qw & \qw & \ghost{U_B} & \qw & \qw & \ghost{U_A}& \qw & \ghost{U_B}& \qw & \qw \\
\lstick{\mid x_n \rangle}& \qw & \qw & \qw & \qw &  \targ{-1} \qwx & \qw  & \qw & \qw  & \qw \qwx & \multigate{1}{U_\kappa} & \qw & \targ \qwx & \qw &\qw \qw \qwx & \qw & \qw \\
\lstick{\mid x_{n+1} \rangle}& \qw & \qw & \qw & \qw &  \qw & \qw  & \qw & \qw & \targ \qwx & \ghost{U_\kappa} & \qw  & \qw & \qw &\targ \qwx &\qw& \qw  \\
\lstick{\mid x_{n+2} \rangle}& \qw & \qw & \qw & \qw &  \qw & \qw  & \qw & \qw & \qw & \targ \qwx & \qw  & \qw & \qw  & \qw& \qw & \qw \\
}
\end{align*}
\caption{ A quantum circuit for the proposed oracle $U_\hbar$ for $\kappa=2$ databases, where $f_\kappa=x_nx_{n+1}$.\label{fig:proposed-oracle-circuit-2fns}}
\end{figure}




\end{frame}




\subsection*{The oracle $U_\hbar$ for more than two databases}

\begin{frame}

\frametitle{Common Entries Oracle Construction: $\kappa$ Oracles}
\begin{figure}[H]
\begin{align*}
\Qcircuit @C=0.25em @R=.25em  {
\lstick{\mid x_0 \rangle}& \qw& \multigate{4}{U_0} & \qw & \multigate{4}{U_1}& \qw & . & . & & \qw & \multigate{4}{U_{\kappa-1}} &  \qw&\qw& \qw&\qw &\multigate{4}{U_0} & \qw & \multigate{4}{U_1}& \qw & . & . & & \qw & \multigate{4}{U_{\kappa-1}}&\qw \\
\lstick{\mid x_1 \rangle}& \qw& \ghost{U_0} & \qw & \ghost{U_1}& \qw & . & . & & \qw & \ghost{U_\kappa-1} &  \qw&\qw& \qw&\qw &\ghost{U_0} & \qw & \ghost{U_1}& \qw & . & . & & \qw & \ghost{U_{\kappa-1}}&\qw \\
.& .&  & . &  & . & . & . & & . &   &  . & . & . & . &  & . &  & . & . & . & & . &  &. \\
.& .&  & . &  & . & . & . & & . &   &  . & . & . & . &  & . &  & . & . & . & & . &  &. \\
\lstick{\mid x_{n-1} \rangle}& \qw& \ghost{U_0} & \qw & \ghost{U_1}& \qw & . & . & & \qw & \ghost{U_{\kappa-1}} &  \qw&\qw& \qw&\qw &\ghost{U_0} & \qw & \ghost{U_1}& \qw & . & . & & \qw & \ghost{U_{\kappa-1}}&\qw \\
\lstick{\ket{x_n}} &\qw & \targ \qwx & \qw & \qw \qwx & \qw & . & . & & \qw & \qw\qwx & \qw & \multigate{4}{U_\kappa} & \qw & \qw  & \targ \qwx & \qw & \qw \qwx & \qw & . & . & & \qw & \qw\qwx & \qw & \\
\lstick{\ket{x_{n+1}}} &\qw & \qw & \qw & \targ \qwx & \qw & . & . & & \qw & \qw\qwx & \qw & \ghost{U_\kappa} & \qw & \qw  & \qw & \qw&\targ \qwx  & \qw & . & . & & \qw & \qw\qwx & \qw & \\
  & . & . & . & . & . & . & . & . & . & \qwx & . &  & . & . & . & . & . & . & . & . & .& . & \qwx & . \\
    & . & . & . & . & . & . & . & . & . & \qwx & . &  & . & . & . & . & . & . & . & . & .& . & \qwx & . \\
\lstick{\ket{x_{n+\kappa-1}}} &\qw & \qw & \qw & \qw & \qw & . & . & & \qw & \targ\qwx & \qw & \ghost{U_\kappa} & \qw & \qw  & \qw & \qw&\qw  & \qw & . & . & & \qw & \targ\qwx & \qw & \\
\lstick{\ket{x_{n+\kappa}}} &\qw & \qw & \qw&\qw & \qw & \qw&\qw & \qw & \qw&\qw & \qw & \targ\qwx  &\qw & \qw & \qw&\qw & \qw & \qw&\qw & \qw & \qw&\qw & \qw & \qw &
}
\end{align*}
\caption{A quantum circuit for the proposed oracle $U_\hbar$ for $\kappa$ functions, where $f_\kappa=x_nx_{n+1}\cdots x_{n+\kappa}$.}
\end{figure}



\end{frame}

\section{The Proposed Algorithm}

\begin{frame}[allowframebreaks]
\frametitle{The Proposed Algorithm}

%The algorithm is carried quantum mechanically as follows:

\begin{algorithm}[H]
\label{1st-algo}
\begin{algorithmic}[1]
\State Construct the oracle $U_\hbar$.
% \If{$\forall U_j$, $\parallel U_j\parallel=n$}
% 	 $e=\kappa$
%\Else  
%	~$e=\Delta$	 	 
%\EndIf
\State Set the quantum register to $\ket{0}^{\otimes n}$ and the extra $\kappa+1$ qubits to $\ket{0}$.
\State Apply the Hadamard gates to the first $n$ qubits to create the uniform superposition $\frac{1}{\sqrt{N}}\sum_{i=0}^{N-1}{\ket{i}}\otimes \ket{0}^{\otimes \kappa+1}$.
\State \iterate Iterate over the following \textcolor{red}{$q_c=\frac{\pi}{2\sqrt{2}}\sqrt{\frac{N}{C}}$} steps:
\begin{enumerate}
\item Apply the oracle $U_\hbar$.
\item Apply the diffusion operator $Y$.%=(H^{\otimes n}\otimes I)(2\ket{0}\bra{0}-I_{n+1})(H^{\otimes n}\otimes I)$.

\end{enumerate}
\State Measure the output.
\end{algorithmic}
\caption{The Proposed Algorithm.}
\end{algorithm}


\begin{figure}[H]
\begin{align*}
\Qcircuit @C=1em @R=.7em {
  \lstick{\ket{0}} & /^n \qw & \gate{H^{\otimes n}} & \multigate{2}{U_{\hbar}} & \multigate{2}{Y}	 & \qw  & \meter \\
  \lstick{\ket{0}}   & /^\kappa  \qw  & \qw &  \ghost{U_\hbar}  &\ghost{Y} & \qw &\qw\\
   \lstick{\ket{0}}   & \qw  & \qw   &  \ghost{U_\hbar} & \ghost{Y}   & \qw & \qw   \gategroup{1}{4}{3}{5}{.7em}{_\}} & 
   \\
   &&&\dstick{\mathcal{O}(\kappa\sqrt{N/C})}& 
 }
\end{align*}
\caption{Quantum circuit for the proposed algorithm.
\label{proposed-algorithm-circuit}}
\end{figure}

\end{frame}



\section{Analysis of the Proposed Algorithm}
\begin{frame}
\frametitle{Analysis of the Proposed Algorithm}

\begin{itemize}

\item \textit{In case of \textcolor{red}{known} number of common matches $C$ between $\kappa$ databases.} The proposed algorithm requires $\mathcal{O}(\kappa\sqrt{N/C})$, where $1\leq C\leq N$. 

\item \textit{In case of \textcolor{red}{unknown} number of common matches between databases}. 
\begin{itemize}
\item An algorithm \cite{quant-count} for estimating the number of matches was presented by Brassard \textit{et al.}, known as \textcolor{red}{quantum counting}.

\item Another algorithm \cite{younes} was presented by Younes \textit{et al.} to search for a match in a database, with \textcolor{red}{unknown number} of matches.

\end{itemize}

\end{itemize}


\end{frame}

\section{Comparison with other Literature}
\begin{frame}
\frametitle{Comparison with other Literature}
\begin{table}[H]
\begin{tabular}{|c|c|c|}
\hline
& Tulsi \cite{tulsi} &Proposed Algorithm \\
\hline
Number of common entries $C$ & $1\leq C\leq 3N/4$ & $1\leq C\leq N$ \\ \hline
Number of databases $\kappa$ & $\kappa=2$ & $\kappa\geq 2$ \\ \hline
Query calls: $\kappa=2,~C=1$ & $\mathcal{O}(\sqrt{N})$ & $\mathcal{O}(\sqrt{N})$ \\ \hline 
Query calls: $\kappa=2,C\geq 2$ & $\mathcal{O}(\sqrt{N/C})$ & $\mathcal{O}(\sqrt{N/C})$  \\ \hline
Query calls: $\kappa>2, C\geq 1$ & NA & $\mathcal{O}(\kappa\sqrt{N/C})$  \\ \hline 


\end{tabular} 

\caption{Comparison between the proposed algorithm and relevant literature.}

\end{table}


\end{frame}


\section{Perspective}
\begin{frame}[allowframebreaks]
\frametitle{Conclusion}

\begin{itemize}

\item Proposed a quantum algorithm to \textcolor{red}{find the common entries} between $\kappa$ databases.

\item Each database uses an \textcolor{red}{oracle to access} its entries.

\item Using the given oracles, we \textcolor{red}{constructed} another oracle that exhibits the \textcolor{red}{behavior of finding} only the common entries between those databases, using \textcolor{red}{entanglement}.


\item \textcolor{red}{Amplitude amplification} algorithm is followed to increase the success probability of finding the common entries.


\item The proposed algorithm \textcolor{red}{requires $\mathcal{O}(\kappa\sqrt{N/C})$} to find the common matches, such that $1\leq C\leq N$.


\item The proposed oracle \textcolor{red}{can be extended} using \cite{quant-count} to \textcolor{red}{count} the number of common entries between any given oracles, or find a match as in \cite{quant-tightbound-for-search,younes}, when the number of common entries $C$ is \textcolor{red}{unknown}.


\item The proposed algorithm can be used to \textcolor{red}{solve a system of binary equations} with \textcolor{red}{no constraints} on the form of the equations contrary to \cite{Schwabe}.


\item Utilizing the proposed oracle with quantum counting, it can be used to measure the \textcolor{red}{Hamming distance} between oracles similar to \cite{Xie2018,hamming-khaled}.

\end{itemize}



\end{frame}


\begin{withoutheadline}
 \begin{frame}
  \vfill
  \centering
  \begin{beamercolorbox}[sep=8pt,center,shadow=true,rounded=true]{title}
    \usebeamerfont{title} Bibliography\par%
  \end{beamercolorbox}
  \vfill
  \end{frame}
\end{withoutheadline}  

\begingroup 
    \setbeamertemplate{headline}{}

\addcontentsline{toc}{Bibliography}

\bibliography{references}
\bibliographystyle{IEEEtran}

\endgroup


\end{document}